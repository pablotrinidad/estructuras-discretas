\documentclass[11pt,letterpaper]{article}

\usepackage[utf8]{inputenc}
\usepackage[spanish,es-nodecimaldot]{babel}
\usepackage{graphicx}
\usepackage{amsmath}
\usepackage{amssymb}
\usepackage{qtree}
\usepackage{color}

\usepackage[top=1in, bottom=1in, left=1in, right=1in]{geometry}


\begin{document}

\begin{titlepage}
    \centering

    {\scshape\LARGE Universidad Nacional Autónoma de México \par}

    \vspace{1cm}
    {\scshape\Large Facultad de Ciencias\par}
    \vspace{1.5cm}

    \begin{center}
        \includegraphics[scale=.1]{../../assets/img/logo.png}
    \end{center}

    \vspace{.8 cm}

    {\LARGE Tarea semanal 01: \par}
    {\huge\bfseries Lenguajes Formales \par}

    \vspace{0.5cm}
    {\large\itshape Pablo A. Trinidad Paz\par}

    \vfill

    Trabajo presentado como parte del curso de \textbf{Estructuras Discretas}
    impartido por la profesora \textbf{Pilar Selene Linares Arévalo}. \par
    \vspace{0.1cm}
    {\large 23 de agosto de 2018\par}
\end{titlepage}

{\large \bfseries Planteamiento \par}

\begin{enumerate}
    \item Considera el conjunto $\mathcal{L}$ que contiene todas las cadenas de
    $a's$ seguidas de $b's$ cuya característica es que siempre aparece al menos
    una $a$ y además, el número de $b's$ es el doble del número de $a's$. Es decir,
        \begin{equation*}
            \mathcal{L} = \{ a^n b^{2n} \mid n > 0\}
        \end{equation*}

    Algunos ejemplos de cadenas que pertenecen al conjunto $\mathcal{L}$ son las
    siguientes: $abb, aabbbb, aaaabbbbbbbb$.

    \begin{enumerate}
        \item (5 pts) Construye una gramática que genere los elementos de $\mathcal{L}$.
        \item (3 pts) Muestra una derivación para la expresión $aaaabbbbbbbb$.
        \item (2 pts) Decide si la cadena $aabb$ es correcta bajo la gramática
        del inciso $a$.
    \end{enumerate}

\end{enumerate}

{\large \bfseries Solución \par}

\begin{enumerate}
    \item
        \begin{enumerate}
            \item Construcción de la gramática para expresiones de $\mathcal{L}$:
                \begin{equation*}
                \begin{split}
                \begin{aligned}
                    S &{}::= P \\
                    1) P &{}::= aPbb \\
                    2) P &{}::= abb
                \end{aligned}
                \end{split}
                \end{equation*}
            \item Derivación de la expresión $aaaabbbbbbbb$: \\
                \begin{center}
                    \Tree [.S
                        [.P
                            {\boxed{a}}
                            [.P
                                {\boxed{a}}
                                [.P
                                    {\boxed{a}}
                                    [.P
                                        {\boxed{abb}}
                                    ]
                                    {\boxed{bb}}
                                ]
                                {\boxed{bb}}
                            ]
                            {\boxed{bb}}
                        ]
                    ]
                \end{center}
                \vspace{.5cm}
                De $S$ se puede llegar a la expresión $aaaabbbbbbbb$ $\therefore$
                $aaaabbbbbbbb$ es una expresión $\in \mathcal{L}$.
                \vspace{5cm}
            \item Decidir si $aabb$ es una expresión que $\in \mathcal{L}$ usando la
            gramática del inciso $a$: \\
            \begin{center}
                \Tree [.S
                    [.P
                        {\boxed{a}}
                        P
                        {\boxed{bb}}
                    ]
                ]
            \end{center}
            \vspace{.5cm}

            Ningún símbolo dentro de $P$ puede reescribir el valor faltante $a$
            y no existe otro árbol de derivación para $aabb \therefore aabb \notin \mathcal{L}$

        \end{enumerate}
\end{enumerate}

\end{document}
