\documentclass[11pt,letterpaper]{article}

\usepackage[utf8]{inputenc}
\usepackage[spanish,es-nodecimaldot]{babel}

\usepackage[top=1in, bottom=1in, left=1in, right=1in]{geometry}

\usepackage{graphicx}

\begin{document}

\begin{titlepage}
    \centering

    {\scshape\LARGE Universidad Nacional Autónoma de México \par}

    \vspace{1cm}
    {\scshape\Large Facultad de Ciencias\par}
    \vspace{1.5cm}

    \begin{center}
        \includegraphics[scale=.1]{../../assets/img/logo.png}
    \end{center}

    \vspace{.8 cm}

    {\LARGE Tarea semanal 01: \par}
    {\huge\bfseries Lenguajes Formales \par}

    \vspace{0.5cm}
    {\large\itshape Pablo A. Trinidad Paz\par}

    \vfill

    Trabajo presentado como parte del curso de \textbf{Estructuras Discretas}
    impartido por la profesora \textbf{Pilar Selene Linares Arévalo}. \par
    \vspace{0.1cm}
    {\large 23 de agosto de 2018\par}
\end{titlepage}

{\large \bfseries Planteamiento \par}

\begin{enumerate}
    \item Considera el conjunto $\mathcal{L}$ que contiene todas las cadenas de
    $a's$ seguidas de $b's$ cuya característica es que siempre aparece al menos
    una $a$ y además, el número de $b's$ es el doble del número de $a's$. Es decir,
        \begin{equation}
            \mathcal{L} = \{ a^n b^{2n} \mid n > 0\}
        \end{equation}

    Algunos ejemplos de cadenas que pertenecen al conjunto $\mathcal{L}$ son las
    siguientes: {\ttfamily abb, aabbbb, aaaabbbbbbbb. \par}

    \begin{enumerate}
        \item (5 pts) Construye una gramática que genere los elementos de $\mathcal{L}$.
        \item (3 pts) Muestra una derivación para la expresión \texttt{aaaabbbbbbbb}.
        \item (2 pts) Decide si la cadena \texttt{aabb} es correcta bajo la gramática
        del inciso $a$.
    \end{enumerate}

\end{enumerate}

{\large \bfseries Solución \par}

\end{document}
