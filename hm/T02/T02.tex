\documentclass[11pt,letterpaper]{article}

\usepackage[utf8]{inputenc}
\usepackage[spanish,es-nodecimaldot]{babel}
\usepackage{amsmath}
\usepackage{amssymb}
\usepackage{graphicx}

\usepackage{mathtools}
\usepackage{tikz}
\usetikzlibrary{trees,positioning}

\usepackage[top=1in, bottom=1in, left=1in, right=1in]{geometry}


\begin{document}

\begin{titlepage}
    \centering

    {\scshape\LARGE Universidad Nacional Autónoma de México \par}

    \vspace{1cm}
    {\scshape\Large Facultad de Ciencias\par}
    \vspace{1.5cm}

    \begin{center}
        \includegraphics[scale=.1]{../../assets/img/logo.png}
    \end{center}

    \vspace{.8 cm}

    {\LARGE Tarea semanal 02: \par}
    {\huge\bfseries Lógica proposicional \par}

    \vspace{0.5cm}
    {\large\itshape Pablo A. Trinidad Paz\par}
    419004279

    \vfill

    Trabajo presentado como parte del curso de \textbf{Estructuras Discretas}
    impartido por la profesora \textbf{Pilar Selene Linares Arévalo}. \par
    \vspace{0.1cm}
    {\large 29 de agosto de 2018\par}
\end{titlepage}

\begin{enumerate}
    \item Encuentra el valor de verdad para las fórmulas generadas por los
    siguientes árboles de derivación, en el estado de las variables $p = 1$,
    $r = 1$ y $q = 0$. \\
    \begin{enumerate}
        \item [a])\\
            \begin{tikzpicture}[nodes={draw,circle},align=center]
                \node {$\to$}
                child {node {$\lor$}
                    child {node {r}}
                    child {node {$\neg$}
                        child {node {p}}
                    }
                }
                child {node {$\neg$}
                    child {node {p}}
                };
            \end{tikzpicture}

            El árbol se reescribe como $r \lor \neg p \to \neg p$ y sustituyendo
            por los valores de $p$ y $r$:
                \begin{equation}
                \begin{split}
                \begin{aligned}
                    r \lor \neg p \to \neg p \\
                    1 \lor \neg 1 \to \neg 1 \\
                    1 \lor 0 \to 0 \\
                    1 \to 0 \\
                    0 \\
                    \therefore r \lor \neg p \to \neg p = 0
                \end{aligned}
                \end{split}
                \end{equation}
    \clearpage
    \item [b])\\
        \begin{tikzpicture}[nodes={draw,circle},align=center]
            \node {$\neg$}
            child {node {$\to$}
                child {node {$\neg$}
                    child {node {$\lor$}
                        child {node {$p$}}
                        child {node {$\land$}
                            child {node {$q$}}
                            child {node {$\neg$}
                                child {node {$p$}}
                            }
                        }
                    }
                }
                child {node {$r$}}
            };
        \end{tikzpicture}

        El árbol se reescribe como $\neg (\neg (p \lor (q \land \neg p)) \to r)$
        y sustituyendo por los valores de $p$, $q$ y $r$:

            \begin{equation}
            \begin{split}
            \begin{aligned}
                \neg (\neg (p \lor (q \land \neg p)) \to r) \\
                \neg (\neg (1 \lor (0 \land \neg 1)) \to 1) \\
                \neg (\neg (1 \lor (0 \land 0)) \to 1) \\
                \neg (\neg (1 \lor 0) \to 1) \\
                \neg (\neg (1) \to 1) \\
                \neg (0 \to 1) \\
                \neg (1) \\
                0 \\
                \therefore \neg (\neg (p \lor (q \land \neg p)) \to r) = 0
            \end{aligned}
            \end{split}
            \end{equation}
    \end{enumerate}

    \clearpage

    \item Utilizando equivalencias lógicas, decide si la siguiente fórmula es
    una tautología, contradicción o contingencia:
        \begin{equation}
        \begin{split}
        \begin{aligned}
            (p \to q) \to (r \to s) \to p \land r \to q \land s
        \end{aligned}
        \end{split}
        \end{equation}

    Aplicamos jerarquía de operaciones para agregar paréntesis a la operación:
        \begin{equation}
        \begin{split}
        \begin{aligned}
            (p \to q) \to (r \to s) \to p \land r \to q \land s \\
            (p \to q) \to (r \to s) \to (p \land r) \to (q \land s) \\
            (p \to q) \to (r \to s) \to ((p \land r) \to (q \land s)) \\
            (p \to q) \to ((r \to s) \to ((p \land r) \to (q \land s)))
        \end{aligned}
        \end{split}
        \end{equation}
    Usando las equivalencias:
        \begin{equation}
        \begin{split}
        \begin{aligned}
            (p \to q) \to ((r \to s) \to ((p \land r) \to (q \land s))) \\
            (p \to q) \to ((r \to s) \to (\neg(p \land r) \lor (q \land s))) \\
            (p \to q) \to ((r \to s) \to ((\neg p \lor \neg r) \lor (q \land s))) \\
            (p \to q) \to (\neg (r \to s) \lor ((\neg p \lor \neg r) \lor (q \land s))) \\
            (p \to q) \to (\neg (\neg r \lor s) \lor ((\neg p \lor \neg r) \lor (q \land s))) \\
            (p \to q) \to ((r \land \neg s) \lor ((\neg p \lor \neg r) \lor (q \land s))) \\
            (p \to q) \to ((r \land \neg s) \lor (\neg p \lor \neg r) \lor (q \land s)) \\
            (\neg p \lor q) \to ((r \land \neg s) \lor (\neg p \lor \neg r) \lor (q \land s)) \\
            \neg (\neg p \lor q) \lor ((r \land \neg s) \lor (\neg p \lor \neg r) \lor (q \land s)) \\
            (p \land \neg q) \lor ((r \land \neg s) \lor (\neg p \lor \neg r) \lor (q \land s)) \\
            (p \land \neg q) \lor (r \land \neg s) \lor (\neg p \lor \neg r) \lor (q \land s) \\
            (p \land \neg q) \lor (r \land \neg s) \lor \neg p \lor \neg r \lor (q \land s) \\
            (\neg p \lor (p \land \neg q)) \lor (\neg r \lor (r \land \neg s)) \lor (q \land s) \\
            ((\neg p \lor p) \land (\neg p \lor \neg q)) \lor (\neg r \lor (r \land \neg s)) \lor (q \land s) \\
            (1 \land (\neg p \lor \neg q)) \lor (\neg r \lor (r \land \neg s)) \lor (q \land s) \\
            (1 \land (\neg p \lor \neg q)) \lor ((\neg r \lor r) \land (\neg r \lor \neg s)) \lor (q \land s) \\
            (1 \land (\neg p \lor \neg q)) \lor (1 \land (\neg r \lor \neg s)) \lor (q \land s) \\
            (\neg p \lor \neg q) \lor (\neg r \lor \neg s) \lor (q \land s) \\
            \neg p \lor \neg q \lor \neg r \lor \neg s \lor (q \land s) \\
            \neg p \lor \neg q \lor \neg r \lor (\neg s \lor (q \land s)) \\
            \neg p \lor \neg q \lor \neg r \lor ((\neg s \lor s) \land(\neg s \lor q)) \\
            \neg p \lor \neg q \lor \neg r \lor (1 \land(\neg s \lor q)) \\
            \neg p \lor \neg q \lor \neg r \lor (\neg s \lor q) \\
            \neg p \lor \neg q \lor \neg r \lor \neg s \lor q \\
            \neg p \lor \neg r \lor \neg s \lor (q \lor \neg q) \\
            \neg p \lor \neg r \lor \neg s \lor 1 \\
            1 \\
            \therefore \text{la fórmula es tautología}
        \end{aligned}
        \end{split}
        \end{equation}
\end{enumerate}

\end{document}
