\documentclass[11pt,letterpaper]{article}

\usepackage[utf8]{inputenc}
\usepackage[spanish,es-nodecimaldot]{babel}
\usepackage{amsmath}
\usepackage{amssymb}
\usepackage{graphicx}

\usepackage[most]{tcolorbox}

\usepackage{mathtools}
\usepackage{tikz}
\usetikzlibrary{trees,positioning}

\usepackage[top=1in, bottom=1in, left=1in, right=1in]{geometry}


\begin{document}

\begin{titlepage}
    \centering

    {\scshape\LARGE Universidad Nacional Autónoma de México \par}

    \vspace{1cm}
    {\scshape\Large Facultad de Ciencias\par}
    \vspace{1.5cm}

    \begin{center}
        \includegraphics[scale=.1]{../../assets/img/logo.png}
    \end{center}

    \vspace{.8 cm}

    {\LARGE Tarea semanal 04: \par}
    {\huge\bfseries Análisis de Argumentos \par}

    \vspace{0.5cm}
    {\large\itshape Pablo A. Trinidad Paz\par}
    419004279

    \vfill

    Trabajo presentado como parte del curso de \textbf{Estructuras Discretas}
    impartido por la profesora \textbf{Pilar Selene Linares Arévalo}. \par
    \vspace{0.1cm}
    {\large 17 de Septiembre de 2018\par}
\end{titlepage}

\begin{enumerate}
    \item Sea $\Gamma$ un conjunto de fórmulas y $\tau$ una tautología.
    Si $\Gamma$ es insatisfacible, ¿como es $\Gamma \cup \{\tau\}$ ?

        \begin{itemize}
            \item El nuevo conjunto de fórmulas $\varphi = \Gamma \cup \{\tau\}$
            sigue siendo insatisfacible porque para que sea satisfacible debe
            existir una interpretación $\mathcal{I}$ tal que $\mathcal{I}(P) = 1$
            para toda $P \in \varphi$ y aunque todos los estados de $\tau$ son modelos,
            sabemos que no existe ningún estado de que satisfaga todas las fórmulas de
            $\Gamma$.
        \end{itemize}

    \item Decide si los siguientes conjuntos de fórmulas son satisfacibles. Justifica.

        \begin{itemize}
            \item $\Gamma_1 = \{p \lor q \lor r, \neg p, \neg q, \neg r\}$

                \begin{equation*} \begin{split}
                    \text{Para probar si el conjunto de fórmulas } \Gamma_1 \\
                    \text{es satisfacible podemos asumirlo y tratar de encontrar} \\
                    \text{los estados de cada variable proposicional.} \\
                \end{split} \end{equation*}
                \begin{equation*} \begin{split} \begin{aligned}
                    \text{1) }&\mathcal{I}(\Gamma_1) = 1 \\
                    \text{2) }&\mathcal{I}(p \lor q \lor r) = 1 \\
                    \text{3) }&\mathcal{I}(\neg p) = 1 \\
                    \text{4) }&\mathcal{I}(\neg q) = 1 \\
                    \text{5) }&\mathcal{I}(\neg r) = 1 \\
                    \text{6) }&\mathcal{I}(p) = 0 \text{ (por \textbf{3})} \\
                    \text{7) }&\mathcal{I}(q) = 0 \text{ (por \textbf{4})} \\
                    \text{8) }&\mathcal{I}(r) = 0 \text{ (por \textbf{5})} \\
                \end{aligned} \end{split} \end{equation*}
                \begin{equation*} \begin{split}
                    \text{Hemos llegado a una contradicción ya que} \\
                    \mathcal{I}(p \lor q \lor r) \text{ no puede evaluarse a 1} \\
                    \text{porque } p, q \text{ y } r \text{ son } 0  \\
                    \therefore \nexists I \, | \, I(P) = 1 \forall P \in \Gamma_1 \\
                    \tcboxmath{\therefore \Gamma_1 \text{ es \textbf{Insatisfacible}}}
                \end{split} \end{equation*}

            \item $\Gamma_2 = \{ \}$
        \end{itemize}
\end{enumerate}

\end{document}
