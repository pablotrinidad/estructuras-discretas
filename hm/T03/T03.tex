\documentclass[10pt,letterpaper]{article}

\usepackage[utf8]{inputenc}
\usepackage[spanish,es-nodecimaldot]{babel}
\usepackage{amsmath}
\usepackage{amssymb}
\usepackage{graphicx}

\usepackage[most]{tcolorbox}

\usepackage{mathtools}
\usepackage{tikz}
\usetikzlibrary{trees,positioning}

\usepackage[top=1in, bottom=1in, left=1in, right=1in]{geometry}


\begin{document}

\begin{titlepage}
    \centering

    {\scshape\LARGE Universidad Nacional Autónoma de México \par}

    \vspace{1cm}
    {\scshape\Large Facultad de Ciencias\par}
    \vspace{1.5cm}

    \begin{center}
        \includegraphics[scale=.1]{../../assets/img/logo.png}
    \end{center}

    \vspace{.8 cm}

    {\LARGE Tarea semanal 03: \par}
    {\huge\bfseries Lógica proposicional \par}

    \vspace{0.5cm}
    {\large\itshape Pablo A. Trinidad Paz\par}
    419004279

    \vfill

    Trabajo presentado como parte del curso de \textbf{Estructuras Discretas}
    impartido por la profesora \textbf{Pilar Selene Linares Arévalo}. \par
    \vspace{0.1cm}
    {\large 6 de Septiembre de 2018\par}
\end{titlepage}

\begin{enumerate}
    \item Decide cuál de los siguientes incisos corresponden a un argumento lógico y
    cuál no. Para el inciso que sí es argumento lógico, identifica sus premisas y
    conclusión. Traduce el argumento a lógica proposicional.

        \begin{enumerate}
            \item \textit{
                Durante mucho tiempo, los astrónomos sospecharon que Europa, una
                de las lunas de Júpiter, albergaba un océano debajo de su superficie
                cubierta de hielo. Tenían razón. La técnica para demostrar la
                existencia del océano europeo ahora ha sido empleada para detectar
                un océano en otro satélite de Júpiter, Ganímides, según un trabajo
                anunciado en el reciente encuentro de la Unión de Geofísica Americana
                en San Francisco, California.
            }\\

            \textbf{No es un argumeto lógico} ya que no existe ninguna sucesión de
            hipótesis con la intensión de validar una conclusión.

            \item \textit{
                La tierra se está calentando. Hay dos principales razones. La primera
                es que la quema de carbón, petróleo y gas natural han aumentado
                considerablemente el dióxido de carbono en la atmósfera. Además, el
                dióxido de carbono retiene el calor. La segunda es que los
                clorofluorocarbonos, que se utilizan en los aparatos de aire
                acondicionado y los refrigeradores, atacan la capa de ozono, en
                consecuencia la tierra queda expuesta a los rayos ultravioleta del sol.
            }\\

            \textbf{Premisas:}
            \begin{itemize}
                \item $p = $ La tierra se está calentando.
                \item $q = $ La quema de carbón, petróleo y gas natural han aumentado
                considerablemente el dióxido de carbono en la atmósfera.
                \item $r = $ El dióxido retiene el calor.
                \item $s = $ Los fluorocarbonos, ..., atacan la capa de ozono.
                \item $t = $ La tierra queda expuesta a los rayos ultravioleta del sol
            \end{itemize}

            \medskip

            \textbf{Desarrollo del argumento:} \\
            Sabemos que la tierra se está calentando es producto de dos cosas,
            por lo tanto podemos decir:
                \begin{equation} \begin{split}
                    P \land Q \implies p
                \end{split} \end{equation}

            Y sabemos que la primera es que existe $q$ pero también que más dióxido
            de carbono implica la retención de calor ($r$), es decir:
                \begin{equation} \begin{split}
                    P = q \land r \\
                    \text{y reemplzando en 1}: \\
                    (q \land r) \land Q \implies p
                \end{split} \end{equation}

            Por último sabemos que $Q$ es la segunda razón de que la tierra se caliente
            ($p$) y esta a su vez se debe a que los fluorocarbonos atacan la capa de ozono
            ($s$) y en consecuencia la tierra queda expuesta a los rayos del sol ($t$), es decir:
                \begin{equation} \begin{split}
                    Q = s \implies t \\
                    \text{y reemplzando en 2}: \\
                    (q \land r) \land (s \implies t) \implies p \\
                    \therefore \text{el argumento se traduce a: } \\
                    \tcboxmath{(q \land r) \land (s \implies t) \implies p}
                \end{split} \end{equation}
        \end{enumerate}

    \clearpage
    \item Sea $\varphi$ una fórmula que es una contradicción. Construye tres fórmulas
    diferentes que sean tautología y que involucren a $\varphi$.
        \begin{enumerate}
            \item [1.-]
                \begin{equation*} \begin{split}
                    \tcboxmath{\varphi \lor \neg \varphi} \\
                    \text{Desarrollo:} \\
                    0 \lor (\neg 0) \\
                    0 \lor 1 \\
                    1 \\
                \end{split} \end{equation*}

            \item [2.-]
                \begin{equation*} \begin{split}
                    \tcboxmath{\neg(\varphi \land \varphi)} \\
                    \text{Desarrollo:} \\
                    \neg (0 \land 0) \\
                    \neg (0) \\
                    1 \\
                \end{split} \end{equation*}

            \item [3.-]
                \begin{equation*} \begin{split}
                    \tcboxmath{\neg(\varphi \land \neg \varphi)} \\
                    \text{Desarrollo:} \\
                    \neg(0 \land (\neg 0)) \\
                    \neg(0 \land 1) \\
                    \neg(0) \\
                    1 \\
                \end{split} \end{equation*}
        \end{enumerate}
\end{enumerate}

\end{document}
