\documentclass[11pt,letterpaper]{article}

\usepackage[utf8]{inputenc}
\usepackage[spanish,es-nodecimaldot]{babel}
\usepackage{amsmath}
\usepackage{amssymb}
\usepackage{graphicx}

\usepackage{multicol}
\usepackage{listings}

\usepackage{tabu}

\usepackage[most]{tcolorbox}

\usepackage{mathtools}
\usepackage{tikz}
\usetikzlibrary{trees,positioning}

\usepackage[top=1in, bottom=1in, left=1in, right=1in]{geometry}


\begin{document}

\begin{titlepage}
    \centering

    {\scshape\LARGE Universidad Nacional Autónoma de México \par}

    \vspace{1cm}
    {\scshape\Large Facultad de Ciencias\par}
    \vspace{1.5cm}

    \begin{center}
        \includegraphics[scale=.1]{../../assets/img/logo.png}
    \end{center}

    \vspace{.8 cm}

    {\LARGE Tarea semanal 07: \par}
    {\huge\bfseries Definiciones recursivas \par}

    \vspace{0.5cm}
    {\large\itshape Pablo A. Trinidad Paz\par}
    419004279

    \vfill

    Trabajo presentado como parte del curso de \textbf{Estructuras Discretas}
    impartido por la profesora \textbf{Pilar Selene Linares Arévalo}. \par
    \vspace{0.1cm}
    {\large 19 de Octubre de 2018\par}
\end{titlepage}

\begin{enumerate}

    \item Da una definición recursiva para el siguiente conjunto: \\
        \begin{equation*} \begin{split} \begin{gathered}
            L = \{a^i b^k | i > 0 \text{ y } k \geq 2i\}
        \end{gathered} \end{split} \end{equation*}

        \textbf{Solución:}
        \begin{itemize}
            \item $abb \in L$
            \item Si $w \in L$, entonces $awbb \in L$
            \item Si $w \in L$, entonces $wb \in L$
        \end{itemize}

    \item Sea $g$ una función que toma un natural $n > 0$ y regresa la lista de naturales
    desde $s(0)$ hasta $n$. Define $g$ recursivamente. \\

        \textbf{Solución:}
        \lstinputlisting[language=Haskell]{Solution.hs}


\end{enumerate}

\end{document}
