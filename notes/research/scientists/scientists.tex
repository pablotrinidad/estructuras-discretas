\documentclass[11pt,letterpaper]{article}

\usepackage[utf8]{inputenc}
\usepackage[spanish,es-nodecimaldot]{babel}
\usepackage{amsmath}
\usepackage{amssymb}
\usepackage{graphicx}

\usepackage[most]{tcolorbox}

\usepackage{mathtools}
\usepackage{tikz}
\usetikzlibrary{trees,positioning}

\usepackage[top=1in, bottom=1in, left=1in, right=1in]{geometry}


\begin{document}

\begin{titlepage}
    \centering

    {\scshape\LARGE Universidad Nacional Autónoma de México \par}

    \vspace{1cm}
    {\scshape\Large Facultad de Ciencias\par}
    \vspace{1.5cm}

    \begin{center}
        \includegraphics[scale=.1]{../../../assets/img/logo.png}
    \end{center}

    \vspace{.8 cm}

    {\LARGE Tarea \par}
    {\huge\bfseries De Morgan, Boole y Shannon \par}

    \vspace{0.5cm}
    {\large\itshape Pablo A. Trinidad Paz\par}
    419004279

    \vfill

    Trabajo presentado como parte del curso de \textbf{Estructuras Discretas}
    impartido por la profesora \textbf{Pilar Selene Linares Arévalo}. \par
    \vspace{0.1cm}
    {\large 27 de Septiembre de 2018\par}
\end{titlepage}

\section{Augustus De Morgan}
    \begin{itemize}
        \item ¿Quién fue? \\
            Fue un matemático y lógico británico que vivió durante el siglo \textit{XVII},
            formuló las \textbf{Leyes de De Morgan} e introdujo el término de
            \textbf{Inducción matemática}.

        \item ¿Cuáles fueron sus aportaciones? \\
            Escribió textos matemáticos como \textit{Elements of Arithmetic (1830)},
            \textit{Penny Cyclopedia (1838)} donde acuñó el término \textit{inducción matemática},
            \textit{Trigonometry and Double Algebra (1849)} donde le dio una interpretación
            geométrica a los números complejos y la lógica formal, entre otros.

            Es mejor conocido por la ley de Morgan la cual dice:
                \begin{equation*} \begin{split} \begin{aligned}
                    \neg (P \lor Q) = \neg P \land \neg Q \\
                    \neg (P \land Q) = \neg P \lor \neg Q \\
                \end{aligned} \end{split} \end{equation*}

        \item ¿Por qué es relevante su trabajo en Ciencias de la Computación? \\

            Por su trabajo junto con George Boole en los fundamentos de la lógica
            simbólica la cuál está muy relacionada actualmente con la semántica de
            los lenguajes de programación. También se ha aplicada como metodología
            para automatizar la verificación de pruebas matemáticas o incluso encontrarlas.
    \end{itemize}

\section{George Boole}
    \begin{itemize}
        \item ¿Quién fue? \\
            Fue un matemático, filósofo y lógico ingés qiuen fue contemporáneo a De Morgan.
            Es mejor conocido como el autor de \textit{The Laws of Thought} el
            cual contiene la \textit{"Boolean algebra"}.
        \item ¿Cuáles fueron sus aportaciones? \\
            \textbf{Boolean algebra:} Es una rama del álgebra donde los valores
            de las variables son valores de verdad (\textit{verdadero} y \textit{falso}).
            La álgebra booleana trabaja con operaciones sobre estos valores como la
            conjunción, disyunción o la negación.
        \item ¿Por qué es relevante su trabajo en Ciencias de la Computación? \\
            La álgebra booleana ha sido fundamental en el desarrollo de componentes
            digitales los cuales sólamente trabajan con los valores de verdadero
            y falso a través de una abstracción de flujo de corriente. Este mismo
            sistema de operaciones es proveído en todos los lenguajes de programación
            \textit{modernos}, además de que también suele ser usado en teoría de
            conjuntos y estadística.
    \end{itemize}

\clearpage
\section{Claude Shannon}
    \begin{itemize}
        \item ¿Quién fue? \\
            Fue un matemático, ingeniero eléctrico y criptógrafo (\textit{cryptographer})
            americano conocido como el padre de la \textbf{teoría de la información}.
        \item ¿Cuáles fueron sus aportaciones? \\
            Haber presentado la \textit{teoría de la información} en su publicación
            del artículo \textit{A Mathematical Theory of Communication}. También es
            conocido por haber fundado la teoría del diseño de circuitos digitales en 1937.
            Cuando tenía 21 años de edad, siendo estudiante de maestría en el MIT,
            escribió su tésis demostrando que las aplicaciones electrónicas
            del \textit{Álgebra booleana} podían construir cualquier relación lógica y numérica.
        \item ¿Por qué es relevante su trabajo en Ciencias de la Computación? \\
            Su trabajo acerca de la teoría de la información estudia la cuantificación,
            almacenamiento y comunicación de la información. En él trata de encontrar
            límites fundamentales al procesamiento de señales y la comunicación de
            operaciones como la \textit{compresión de datos}. Su trabajo es relevante
            en el área porque en áreas desde la invención del \textit{compact disc}
            hasta el plantamiento de los protocólos para el desarrollo del internet.
    \end{itemize}

\end{document}
