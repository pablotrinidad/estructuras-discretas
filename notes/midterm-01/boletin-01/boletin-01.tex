\documentclass[10pt,letterpaper]{article}

\usepackage[utf8]{inputenc}
\usepackage[spanish,es-nodecimaldot]{babel}
\usepackage{amsmath}
\usepackage{amssymb}
\usepackage{graphicx}

\usepackage[most]{tcolorbox}

\usepackage{multicol}

\usepackage{mathtools}
\usepackage{forest}
\usepackage{tikz}
\usetikzlibrary{trees,positioning}

\usepackage[top=1in, bottom=1in, left=1in, right=1in]{geometry}


\begin{document}

\begin{titlepage}
    \centering

    {\scshape\LARGE Universidad Nacional Autónoma de México \par}

    \vspace{1cm}
    {\scshape\Large Facultad de Ciencias\par}
    \vspace{1.5cm}

    \begin{center}
        \includegraphics[scale=.1]{../../../assets/img/logo.png}
    \end{center}

    \vspace{.8 cm}

    {\LARGE Boletín de Ejercicios 01: \par}
    {\huge\bfseries Primer parcial \par}

    \vspace{0.5cm}
    {\large\itshape Pablo A. Trinidad Paz\par}
    419004279

    \vfill

    Trabajo presentado como parte del curso de \textbf{Estructuras Discretas}
    impartido por la profesora \textbf{Pilar Selene Linares Arévalo}. \par
    \vspace{0.1cm}
    {\large 12 de Septiembre de 2018\par}
\end{titlepage}


\begin{enumerate}
    \item Considera la siguiente gramática
        \begin{equation*} \begin{split} \begin{aligned}
            &S :: = E \\
            &E :: = \downarrow E \uparrow \\
            &E :: = \bigcirc E \\
            &E :: = \uparrow \uparrow E \\
            &E :: = \square \alpha \\
            &E :: = \delta
        \end{aligned} \end{split} \end{equation*}

        \begin{enumerate}
            \item Construye una derivación correspondiente a la cadena
                  $\uparrow \uparrow \downarrow \bigcirc \uparrow \uparrow \square \alpha \uparrow$.

                \begin{center}
                    \begin{tikzpicture}[nodes={draw,circle}]
                        \node {$S$}
                        child {node {$E$}
                            child {node {$\uparrow \uparrow$}}
                            child {node {$E$}
                                child {node {$\downarrow$}}
                                child {node {$E$}
                                    child {node {$\bigcirc$}}
                                    child {node {$E$}
                                        child {node {$\uparrow \uparrow$}}
                                        child {node {$E$}
                                            child {node {$\square\alpha$}}
                                        }
                                    }
                                }
                                child {node {$\uparrow$}}
                            }
                        };
                    \end{tikzpicture}
                \end{center}

            \item Da el árbol que corresponde a la expresión $\bigcirc \downarrow \square \alpha \uparrow$.
                \begin{center}
                    \begin{tikzpicture}[nodes={draw, circle}]
                        \node {$S$}
                        child {node {$E$}
                            child {node {$\bigcirc$}}
                            child {node {$E$}
                                child {node {$\downarrow$}}
                                child {node {$E$}
                                    child {node {$\square\alpha$}}
                                }
                                child {node {$\uparrow$}}
                            }
                        };
                    \end{tikzpicture}
                \end{center}

            \clearpage
            \item ¿La cadena $\uparrow \uparrow \uparrow \bigcirc \delta$ está bien formada?
                  Justifique su respuesta.
                \begin{multicols}{2}
                    \begin{center}
                        \begin{tikzpicture}[nodes={draw, circle}]
                            \node {$S$}
                                child {node {$E$}
                                    child {node {$\uparrow \uparrow$}}
                                    child {node {$E$}}
                            };
                        \end{tikzpicture}
                    \end{center}
                    \begin{center}
                        \begin{tikzpicture}[nodes={draw, circle}]
                            \node {$S$}
                                child {node {$E$}
                                    child {node {$\uparrow \uparrow$}}
                                    child {node {$E$}
                                        child {node {$\bigcirc$}}
                                        child {node {$E$}
                                            child {node {$\delta$}}
                                        }
                                    }
                            };
                        \end{tikzpicture}
                    \end{center}
                \end{multicols}
                La cadena no está bien formulada ya que a partir de los únicos dos posibles
                árboles de derivación fue imposible llegar a la cadena final.
        \end{enumerate}
\end{enumerate}


\end{document}
